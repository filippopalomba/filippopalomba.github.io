%%%%%%%%%%%%%%%%%%%%%%%%%%%%%%%%%%%%%%%%%
% Medium Length Professional CV
% LaTeX Template
% Version 2.0 (8/5/13)
%
% This template has been downloaded from:
% http://www.LaTeXTemplates.com
%
% Original author:
% Trey Hunner (http://www.treyhunner.com/)
%
% Important note:
% This template requires the resume.cls file to be in the same directory as the
% .tex file. The resume.cls file provides the resume style used for structuring the
% document.
%
%%%%%%%%%%%%%%%%%%%%%%%%%%%%%%%%%%%%%%%%%

%----------------------------------------------------------------------------------------
%	PACKAGES AND OTHER DOCUMENT CONFIGURATIONS
%----------------------------------------------------------------------------------------

\documentclass{resume} % Use the custom resume.cls style

\usepackage[left=1in,top=1in,right=1in,bottom=1.2in]{geometry} % Document margins
\usepackage{color}  
\definecolor{airforceblue}{rgb}{0.20, 0.30, 1}
\definecolor{lightgrey}{rgb}{0.3,0.4,0.3}
\definecolor{porange}{RGB}{204,102,0}
\usepackage{hyperref}
\usepackage{lscape}
\usepackage{comment}
\hypersetup{
   colorlinks=true, %set true if you want colored links
   linktoc=all,        %set to all if you want both sections and subsections linked
   linkcolor=porange,  %choose some color if you want links to stand out
   urlcolor  = porange,
   citecolor = black,
}
\name{  \LARGE Filippo Palomba} % Your name

\address{ \href{mailto:fpalomba@princeton.edu}{fpalomba@princeton.edu} \\ \href{https://www.fpalomba.com}{www.fpalomba.com}
\\ 609-356-8011 }

\begin{document}

\vspace{0.3cm}

%\begin{tabular}{@{\hspace{0cm}}llll}
%{\bf Placement Director} & Steve Redding & \href{mailto:reddings@princeton.edu}{reddings@princeton.edu} & 609-258-4016 \smallskip \\ {\bf Graduate Administrator} & Laura Hedden & \href{mailto:lhedden@princeton.edu}{lhedden@princeton.edu} & 609-258-4006
%\end{tabular}

%\vspace{0.3cm}

%----------------------------------------------------------------------------------------
%	CONTACT
%----------------------------------------------------------------------------------------

%\begin{rSection}{Office Contact Information}

%Julis Romo Rabinowitz Building \smallskip \\
%Department of Economics \smallskip \\
%Princeton University \smallskip \\
%Princeton, NJ 08544

%\end{rSection}

%----------------------------------------------------------------------------------------
%	GRADUATE STUDIES
%----------------------------------------------------------------------------------------

\begin{rSection}{Graduate Studies}

{\bf Princeton University} \medskip \hfill {\em 2020-present} \\
PhD Candidate in Economics \smallskip \\
%Dissertation: \emph{``Essays on Identification in Macroeconomics''} \smallskip \\
Expected Completion Date: June 2025 \medskip

%{\sc References}

%\vspace{0.3cm}

%\begin{tabular}{@{\hspace{0cm}}ll}
%Professor Gianluca Violante \hspace{1cm} & Professor Mark Watson \\ Department of Economics & Department of Economics \\ Princeton University & Princeton University \\ 609-258-4003 & 609-258-4811 \\ \href{mailto:violante@princeton.edu}{violante@princeton.edu} & \href{mailto:mwatson@princeton.edu}{mwatson@princeton.edu} \bigskip \\ Professor Benjamin Moll \hspace{1cm} & Professor Mikkel Plagborg-M{\o}ller \\ Department of Economics & Department of Economics \\ Princeton University & Princeton University \\ 609-258-0329 & 617-893-0359 \\ \href{mailto:moll@princeton.edu}{moll@princeton.edu} & \href{mailto:mikkelpm@princeton.edu}{mikkelpm@princeton.edu}
%\end{tabular}

\end{rSection}

%----------------------------------------------------------------------------------------
%	PRIOR EDUCATION
%----------------------------------------------------------------------------------------

\begin{rSection}{Prior Education}

{\bf Bocconi University} \medskip \hfill {\em 2016-2018} \\
M.Sc. in Economic and Social Sciences

{\bf University of Padua} \medskip \hfill {\em 2013-2016} \\
B.Sc. in Economics and Management

\end{rSection}

%----------------------------------------------------------------------------------------
%	FIELDS
%----------------------------------------------------------------------------------------

\begin{rSection}{Fields}
\begin{tabular}{@{\hspace{0cm}}ll}
{\sc Primary} & Econometrics, Causal Inference \medskip \\ {\sc Secondary} & Program Evaluation
\end{tabular}
\end{rSection}

%----------------------------------------------------------------------------------------
%	PUBLICATIONS
%----------------------------------------------------------------------------------------

%\begin{rSection}{Publications (including accepted \& forthcoming)}

%\begin{enumerate}

%\item \href{https://scholar.princeton.edu/sites/default/files/ckwolf/files/mp_id_201907.pdf}{``SVAR (Mis-)Identification and the Real Effects of Monetary Policy Shocks."} 2019. \textit{Forthcoming at American Economic Journal: Macroeconomics.}

%\end{enumerate}

%\end{rSection}

%----------------------------------------------------------------------------------------
%	WORKING PAPERS
%----------------------------------------------------------------------------------------

%\begin{rSection}{Job Market Paper} 

%\href{https://scholar.princeton.edu/sites/default/files/ckwolf/files/missing_intercept.pdf}{``The Missing Intercept: A Demand Equivalence Approach."} 2019.
%``The Missing Intercept: A Demand Equivalence Approach." 2019.

%{\em I prove that, in a broad class of structural macro models, shocks to private consumption demand elicit the same general equilibrium price responses as changes in aggregate public spending. This demand equivalence result implies that the aggregate effects of a large family of consumption demand shocks can be estimated in two steps. First, a researcher uses cross-sectional heterogeneity in shock exposure to recover the partial equilibrium effect on consumption demand. Second, indirect general equilibrium effects -- the ``missing intercept'' of the micro regression -- are, by demand equivalence, equal to the response of consumption to changes in aggregate public spending. I apply this method to deficit-financed tax rebates, and find (i) a large partial equilibrium response, but (ii) a fiscal multiplier of one and so little further general equilibrium feedback to private spending. Any calibrated structural model that instead implies a non-zero general equilibrium intercept either breaks demand equivalence or features fiscal multipliers far from one. Finally, I show that equivalence extends to generic investment demand shocks, and use it to estimate the effects of bonus depreciation stimulus.}

%\end{rSection}

\begin{rSection}{Working Papers} 

\begin{enumerate}

\item \href{https://filippopalomba.github.io/docs/Cattaneo-Feng-Palomba-Titiunik_2022_SAMT.pdf}{``Uncertainty Quantification in Synthetic Controls with Staggered Treatment Adoption"}. With Matias D. Cattaneo, Yingjie Feng, and Rocio Titiunik. 2022. \textit{working paper}.

\item \href{https://filippopalomba.github.io/docs/Cattaneo-Feng-Palomba-Titiunik_2022_scpi.pdf}{``scpi: Uncertainty Quantification for Synthetic Control Estimators"}. With Matias D. Cattaneo, Yingjie Feng, and Rocio Titiunik. 2022. \textit{working paper}.

\item \href{https://filippopalomba.github.io/docs/Cingano-Palomba-Pinotti-Rettore_2022_subsidies.pdf}{``Making Subsidies Work: Rules vs. Discretion"}. With Federico Cingano, Paolo Pinotti, and Enrico Rettore. 2022. \textit{working paper}.

\item \href{https://filippopalomba.github.io/docs/Goerlach-Palomba_2023.pdf}{``The Selection of Return Migrants: Some Evidence and a Model-based Analysis".} With Joseph-Simon G{\"o}rlach. forthcoming. Prepared for the \textit{World Scientific Encyclopaedia of Global Migration, Volume 3} by World Scientific Publishers.
\end{enumerate}

\end{rSection}

\begin{rSection}{Work in Progress} 

\begin{enumerate}

\item \textit{``Fixed Effects in Climate Econometrics"}. With Thomas Bearpark. 2022.

\item \textit{``The Impact of TV on Education: Evidence from Italy"}. With Chiara Motta. 2022.
\end{enumerate}

\end{rSection}

%----------------------------------------------------------------------------------------
%	RESEARCH EXPERIENCE
%----------------------------------------------------------------------------------------

\begin{rSection}{Research Experience}
\begin{tabular}{@{\hspace{0cm}}ll}
2022 & Visiting Researcher, Bank of Italy \\
2021 & Research Assistant to Prof. Matias Cattaneo \\
2019-20 & Visiting Researcher, European Central Bank \\
2018-19 & Research Assistant to Prof. Paolo Pinotti \\
2017 & Research Assistant to Joseph-Simon G\"{o}rlach
\end{tabular}
\end{rSection}

%----------------------------------------------------------------------------------------
%	TEACHING EXPERIENCE
%----------------------------------------------------------------------------------------

\begin{rSection}{Teaching}
\begin{tabular}{@{\hspace{0cm}}ll}
{\it Princeton} & ECO 202: Statistics and Data Analysis for Economics (TA, Fall 2022) \medskip \\
& ECO 302: Undergraduate Econometrics (TA, Fall 2022) \bigskip \\
{\it Bocconi} & ECO 30066: Undergraduate Macroeconomics (TA, Spring 2019)
\end{tabular}
\end{rSection}

%----------------------------------------------------------------------------------------
%	PROFESSIONAL ACTIVITIES
%----------------------------------------------------------------------------------------

%\begin{rSection}{Professional Activities}

%\textbf{Presentations and Seminars (including scheduled)}

%\begin{tabular}{lp{14cm}}
%2020 & University of Pennsylvania, Chicago Booth, University of Chicago, MIT, Stanford University, UCLA, Columbia University, University of Pennsylvania Wharton \medskip \\
%2019 & Princeton University, European Central Bank, Philadelphia Federal Reserve Bank, University of Cologne, NBER Summer Institute, Norges Bank, Minneapolis Federal Reserve Bank, Chicago Federal Reserve Bank \medskip \\
%2018 & European Central Bank, Philadelphia Federal Reserve Bank, AEA Meetings, University of Mannheim, European Meetings of the Econometric Society \medskip \\
%2017 & Cambridge University, Bundesbank, ifo Institute \bigskip
%\end{tabular}

%{ \bf Refereeing}

%\emph{American Economic Review}, \emph{American Economic Journal: Macroeconomics}, \emph{ECB Working Paper Series}, \emph{Econometrica}, \emph{Journal of Business \& Economic Statistics}, \emph{Journal of Political Economy}, \emph{Review of Economics and Statistics}, \emph{Review of Economic Studies}
%\end{rSection}

%----------------------------------------------------------------------------------------
%	AWARDS
%----------------------------------------------------------------------------------------

\begin{rSection}{Honors, Scholarships, Fellowships, and Grants}

Research Fellowship, Bank of Italy \medskip \hfill{\em 2022} \\
Marco Fanno Scholarship, UniCredit Foundation \medskip \hfill{\em 2020-2021} \\
Princeton Graduate Economics Fellowship \medskip \hfill { \em 2020--2025} \\
Oustanding distinction for ``Giorgio Mortara" scholarship, Bank of Italy \medskip \hfill {\em 2020} \\
ISU Grant \medskip \hfill {\em 2013--2018}

\end{rSection}

%----------------------------------------------------------------------------------------
%	COMPUTER SKILLS
%----------------------------------------------------------------------------------------

\begin{rSection}{Computer Skills}
MATLAB, R, Python, Stata, Julia, QGIS, EViews, SQL
\end{rSection}

%----------------------------------------------------------------------------------------
%	LANGUAGES
%----------------------------------------------------------------------------------------

\begin{rSection}{Languages}
Italian (native), English
\end{rSection}

\vspace{0.5cm}

\hfill {\em Last updated: October 2022}

\end{document}